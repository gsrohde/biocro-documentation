%\VignetteIndexEntry{Introduction to BioCro}
%\VignettePackage{BioCro}
%\VignetteEngine{knitr::knitr}

\documentclass[12pt]{article}\usepackage[]{graphicx}\usepackage[]{color}
% maxwidth is the original width if it is less than linewidth
% otherwise use linewidth (to make sure the graphics do not exceed the margin)
\makeatletter
\def\maxwidth{ %
  \ifdim\Gin@nat@width>\linewidth
    \linewidth
  \else
    \Gin@nat@width
  \fi
}
\makeatother

\definecolor{fgcolor}{rgb}{0.345, 0.345, 0.345}
\newcommand{\hlnum}[1]{\textcolor[rgb]{0.686,0.059,0.569}{#1}}%
\newcommand{\hlstr}[1]{\textcolor[rgb]{0.192,0.494,0.8}{#1}}%
\newcommand{\hlcom}[1]{\textcolor[rgb]{0.678,0.584,0.686}{\textit{#1}}}%
\newcommand{\hlopt}[1]{\textcolor[rgb]{0,0,0}{#1}}%
\newcommand{\hlstd}[1]{\textcolor[rgb]{0.345,0.345,0.345}{#1}}%
\newcommand{\hlkwa}[1]{\textcolor[rgb]{0.161,0.373,0.58}{\textbf{#1}}}%
\newcommand{\hlkwb}[1]{\textcolor[rgb]{0.69,0.353,0.396}{#1}}%
\newcommand{\hlkwc}[1]{\textcolor[rgb]{0.333,0.667,0.333}{#1}}%
\newcommand{\hlkwd}[1]{\textcolor[rgb]{0.737,0.353,0.396}{\textbf{#1}}}%
\let\hlipl\hlkwb

\usepackage{framed}
\makeatletter
\newenvironment{kframe}{%
 \def\at@end@of@kframe{}%
 \ifinner\ifhmode%
  \def\at@end@of@kframe{\end{minipage}}%
  \begin{minipage}{\columnwidth}%
 \fi\fi%
 \def\FrameCommand##1{\hskip\@totalleftmargin \hskip-\fboxsep
 \colorbox{shadecolor}{##1}\hskip-\fboxsep
     % There is no \\@totalrightmargin, so:
     \hskip-\linewidth \hskip-\@totalleftmargin \hskip\columnwidth}%
 \MakeFramed {\advance\hsize-\width
   \@totalleftmargin\z@ \linewidth\hsize
   \@setminipage}}%
 {\par\unskip\endMakeFramed%
 \at@end@of@kframe}
\makeatother

\definecolor{shadecolor}{rgb}{.97, .97, .97}
\definecolor{messagecolor}{rgb}{0, 0, 0}
\definecolor{warningcolor}{rgb}{1, 0, 1}
\definecolor{errorcolor}{rgb}{1, 0, 0}
\newenvironment{knitrout}{}{} % an empty environment to be redefined in TeX

\usepackage{alltt}

\usepackage{amsmath}
\usepackage{bm}
\usepackage{siunitx}
\usepackage[american]{babel}
\usepackage[authoryear,round]{natbib}
\usepackage{hyperref}
\bibliographystyle{plainnat}
  
\title{Simulation and Parameter Estimation for Biomass Crops}
\author{Fernando E. Miguez\\Department of Agronomy\\
  Iowa State University}
\IfFileExists{upquote.sty}{\usepackage{upquote}}{}
\begin{document}








\setkeys{Gin}{width=\textwidth}
\newcommand{\code}[1]{\texttt{\small{#1}}}
\newcommand{\package}[1]{\textsf{\small{#1}}}
\maketitle

\noindent\emph{\textbf{Note:} This is an outdated version of the
BioCro manual.  It refers to R functions in the BioCro package that no
longer exist.  For this reason, all evaluation of R code has been
turned off by setting ``eval=FALSE" globally.  To see a PDF of the
original form of this vignette, visit
\url{http://biocro.r-forge.r-project.org/BioCro-Manual.pdf}.}

\begin{abstract}
  Simulation and parameter estimation of photosynthesis and crop
  growth. The interest in developing this model is to be able to
  efficiently perform simulations of photosynthesis and crop
  growth. Since often this requires running a model multiple times, R
  provides a nice environment for optimization and plotting. The
  package also has a soil carbon and nitrogen model based on the
  Century model and a simple multilayered water soil model. As with
  many crop models the objective is to improve our understanding of
  productivity and carbon, water and nitrogen cycles in
  agro-ecosystems.
\end{abstract}

\tableofcontents



\section{Introduction}

The package \package{BioCro} started as a way to continue work on the
ideas developed in the WIMOVAC model. WIMOVAC was developed by Stephen
Long and Steve Humphries
\url{http://www.life.illinois.edu/plantbio/wimovac/} and there have been
several publications using this model. I have used the model for some
initial efforts at modeling \emph{Miscanthus} $\times$ \emph{giganteus}.

This vignette was created using BioCro package version
.

This version of \package{BioCro} has had contributions from Joe
Iverson, Deepak Jaiswal, David LeBauer and Dandan Wang and possibly
others.  Currently the capabilities of the model are growing with more
species and algorithms being added.

\section{Carbon: Leaf-level Photosynthesis}
 
There is a large range in the complexity of models that have been used
to simulate photosynthesis. \package{BioCro} offers functions
\code{c4photo} and \code{c3photo}. Both functions take as minimum
input radiation (PAR $\si{\micro\mol.\meter^{-2}.\second^{-1}}$),
temperature (Celsius) and relative humidity (0--1).

Since WIMOVAC originated as a photosynthesis model we can start with a
simple example. For the C$_4$ photosynthesis model the best reference
is Collatz et al. (1992). 

\begin{knitrout}
\definecolor{shadecolor}{rgb}{0.969, 0.969, 0.969}\color{fgcolor}\begin{kframe}
\begin{alltt}
\hlkwd{c4photo}\hlstd{(}\hlnum{1500}\hlstd{,}\hlnum{25}\hlstd{,}\hlnum{0.7}\hlstd{)}
\end{alltt}
\end{kframe}
\end{knitrout}

This example shows that with PAR 1500, temperature of 25 and relative
humidity of 0.7 (70\%) as inputs we get simulation of CO$_2$
assimilation (\code{Assim}), stomatal conductance (\code{Gs}) and the
intercellular CO$_2$ (\code{Ci}). For units and other details see
\code{?c4photo}. Another model available for simulating C4
photosynthesis is \code{eC4photo}, see docs for details. This model
has not been used much. In \code{c4photo} the computation is carried
out in compiled \code{C}, but there is a pure R function
\code{c4photoR} which might be useful for understanding the
calculations.

There is an equivalent function \texttt{c3photo} which is closely
based on the c3 photosynthesis model described in WIMOVAC. Notice that
the parameters and interpretation are different from the
\code{c4photo} function. 



% |

\subsection{Temperature response of C$_4$ photosynthesis}

The shape of the temperature response to photosynthesis can be
modified from the default values.

\begin{knitrout}
\definecolor{shadecolor}{rgb}{0.969, 0.969, 0.969}\color{fgcolor}\begin{kframe}
\begin{alltt}
\hlstd{idat} \hlkwb{<-} \hlkwd{data.frame}\hlstd{(}\hlkwc{Qp}\hlstd{=}\hlnum{1500}\hlstd{,} \hlkwc{temp} \hlstd{=} \hlnum{0}\hlopt{:}\hlnum{50}\hlstd{,} \hlkwc{rh} \hlstd{=} \hlnum{0.7}\hlstd{)}
\hlstd{res} \hlkwb{<-} \hlkwd{c4photo}\hlstd{(idat}\hlopt{$}\hlstd{Qp, idat}\hlopt{$}\hlstd{temp, idat}\hlopt{$}\hlstd{rh)}
\hlstd{res2} \hlkwb{<-} \hlkwd{c4photo}\hlstd{(idat}\hlopt{$}\hlstd{Qp, idat}\hlopt{$}\hlstd{temp, idat}\hlopt{$}\hlstd{rh,} \hlkwc{uppertemp} \hlstd{=} \hlnum{45}\hlstd{,} \hlkwc{lowertemp} \hlstd{=} \hlnum{10}\hlstd{)}
\hlkwd{xyplot}\hlstd{(res}\hlopt{$}\hlstd{Assim} \hlopt{+} \hlstd{res2}\hlopt{$}\hlstd{Assim} \hlopt{~} \hlstd{idat}\hlopt{$}\hlstd{temp,} \hlkwc{type} \hlstd{=} \hlstr{'o'}\hlstd{,}
       \hlkwc{xlab} \hlstd{=} \hlstr{"Temperature"}\hlstd{,} \hlkwc{ylab} \hlstd{=} \hlstr{"Assimilation"}\hlstd{,}
       \hlkwc{key} \hlstd{=} \hlkwd{simpleKey}\hlstd{(}\hlkwd{c}\hlstd{(}\hlstr{'default'}\hlstd{,}\hlstr{'U = 45, L = 10'}\hlstd{)))}
\end{alltt}
\end{kframe}
\end{knitrout}

Reducing the first value (\code{lowertemp}) reduces the level of
photosynthesis at low temperatures and increasing the second value
(\code{uppertemp}) results in higher photosynthesis at high
temperatures.

\subsubsection{Effect of stress on photosynthesis}

Besides some of the typical parameters for both functions there is the
option of including stress. The argument is \code{StomWS}. The stress
can be applied to stomatal conductance (default) or $V_{max}$. 



% | 



\subsection{Estimating photosynthesis parameters}

The \code{c4photo} function has three main parameters that can be
adjusted for different species or environments. These three parameters
are $V_{max}$ (light-saturated rate of photosynthesis which for C$_4$
plants is equivalent to the maximum rate of carboxylation,
\si{\micro\mole\per\square\meter\per\second}), $alpha$ (quantum
efficiency, \si{\mole\per\mole}) and dark respiration (net CO$_2$
exchange at zero light, \si{\micro\mole\per\square\meter\per\second}).

Options for adjusting the parameters for simulation of biomass crops:

\begin{itemize}
\item Use values from the literature
\item Optimize parameter values based on observed data
\end{itemize}
      
Using values from the literature can work well most of the time, but
the source of the values should be carefully assessed. For example, an
important consideration is whether the values were reported at the
same reference temperature. In BioCro these values are assumed to be
at \SI{25}{\degreeCelsius}. Another consideration is that different models might have
different parameterization of photosynthesis and what might work well
in a different model might not work well in BioCro.
      
For optimization of these parameters based on observed data BioCro
offers the ability to optimize these three parameters using different
techniques. The first one is based on the \code{optim} function in R
which will minimize the residual sum of squares (obs - sim). The first
column has the curve ID, the second one the treatment which in this
case is either Miscanthus (mxg) or switchgrass (swg). The third column
is CO$_2$ assimilation, then photosynthetic active radiation in
(PARin), temperature (Tleaf) and relative humidity (RH\_S).
      
\begin{knitrout}
\definecolor{shadecolor}{rgb}{0.969, 0.969, 0.969}\color{fgcolor}\begin{kframe}
\begin{alltt}
\hlkwd{data}\hlstd{(aq)}
\hlkwd{head}\hlstd{(aq)}
\hlstd{lattice}\hlopt{::}\hlkwd{xyplot}\hlstd{(A} \hlopt{~} \hlstd{PARi} \hlopt{|} \hlstd{trt,} \hlkwc{data}\hlstd{=aq,} \hlkwc{ylab}\hlstd{=}\hlstr{'CO2 uptake'}\hlstd{,} \hlkwc{xlab}\hlstd{=}\hlstr{'Quantum flux'}\hlstd{)}
\end{alltt}
\end{kframe}
\end{knitrout}

We could find out what the `best' parameters are for the first curve.

\begin{knitrout}
\definecolor{shadecolor}{rgb}{0.969, 0.969, 0.969}\color{fgcolor}\begin{kframe}
\begin{alltt}
\hlstd{curve1} \hlkwb{<-} \hlkwd{subset}\hlstd{(aq, ID} \hlopt{==} \hlnum{1}\hlstd{)}
\end{alltt}
\end{kframe}
\end{knitrout}

\begin{knitrout}
\definecolor{shadecolor}{rgb}{0.969, 0.969, 0.969}\color{fgcolor}\begin{kframe}
\begin{alltt}
\hlstd{op} \hlkwb{<-} \hlkwd{Opc4photo}\hlstd{(curve1[,}\hlnum{3}\hlopt{:}\hlnum{6}\hlstd{])}
\hlstd{op}
\end{alltt}
\end{kframe}
\end{knitrout}

By default only $V_{max}$ and $alpha$ are optimized. $Rd$ can also be optimized
by setting the optimization level to 2.

\begin{knitrout}
\definecolor{shadecolor}{rgb}{0.969, 0.969, 0.969}\color{fgcolor}\begin{kframe}
\begin{alltt}
\hlstd{op} \hlkwb{<-} \hlkwd{Opc4photo}\hlstd{(curve1[,}\hlnum{3}\hlopt{:}\hlnum{6}\hlstd{],} \hlkwc{op.level}\hlstd{=}\hlnum{2}\hlstd{)}
\hlstd{op}
\end{alltt}
\end{kframe}
\end{knitrout}


There is also a plotting function for the \code{op} object which is of class
``Opc4photo''. This allows for examination of residuals and also comparing
observed and simulated.

\begin{knitrout}
\definecolor{shadecolor}{rgb}{0.969, 0.969, 0.969}\color{fgcolor}\begin{kframe}
\begin{alltt}
\hlkwd{plot}\hlstd{(op)}
\hlkwd{plot}\hlstd{(op,} \hlkwc{plot.kind}\hlstd{=}\hlstr{"OandF"}\hlstd{,} \hlkwc{type}\hlstd{=}\hlstr{"o"}\hlstd{)}
\end{alltt}
\end{kframe}
\end{knitrout}

The previous example was for optimizing and analyzing a single curve. If we want
to optimize several curves in parallel then the \code{mOpc4photo} function is
available. For the following example we need to also supply a column with
ambient CO$_2$ levels (in this case 390 ppm).

\begin{knitrout}
\definecolor{shadecolor}{rgb}{0.969, 0.969, 0.969}\color{fgcolor}\begin{kframe}
\begin{alltt}
\hlstd{aq2} \hlkwb{<-} \hlkwd{data.frame}\hlstd{(aq[,}\hlopt{-}\hlnum{2}\hlstd{],} \hlkwc{Catm}\hlstd{=}\hlnum{390}\hlstd{)}
\hlstd{mop} \hlkwb{<-} \hlkwd{mOpc4photo}\hlstd{(aq2,} \hlkwc{verbose}\hlstd{=}\hlnum{TRUE}\hlstd{)}
\hlstd{mop}
\end{alltt}
\end{kframe}
\end{knitrout}

As with the previous example there are plotting methods. In this case the
plotting functions are useful for visualizing the mean point estimate and the
confidence intervals. 

\begin{knitrout}
\definecolor{shadecolor}{rgb}{0.969, 0.969, 0.969}\color{fgcolor}\begin{kframe}
\begin{alltt}
\hlkwd{plot}\hlstd{(mop)}
\hlkwd{plot}\hlstd{(mop,} \hlkwc{parm}\hlstd{=}\hlstr{"alpha"}\hlstd{)}
\end{alltt}
\end{kframe}
\end{knitrout}

As with the \code{Opc4photo} the optimization level can be changed to also
optimize $Rd$.

\begin{knitrout}
\definecolor{shadecolor}{rgb}{0.969, 0.969, 0.969}\color{fgcolor}\begin{kframe}
\begin{alltt}
\hlstd{mop2} \hlkwb{<-} \hlkwd{mOpc4photo}\hlstd{(aq2,} \hlkwc{verbose}\hlstd{=}\hlnum{TRUE}\hlstd{,} \hlkwc{op.level}\hlstd{=}\hlnum{2}\hlstd{)}
\hlstd{mop2}
\end{alltt}
\end{kframe}
\end{knitrout}

Yet another approach to optimizing photosynthetic parameters is to use a
Bayesian approach where a prior distribution for the parameters can be
specified. This approach might work better when there is limited data or when
the previous approaches fail. The results are very similar to the example using
\code{Opc4photo} for a single curve. In particular because in this case very
diffuse priors were specified. The scale can be increased to reduce the
acceptance rate. The function can also be run more than once to check the
results.

\begin{knitrout}
\definecolor{shadecolor}{rgb}{0.969, 0.969, 0.969}\color{fgcolor}\begin{kframe}
\begin{alltt}
\hlstd{op.mc1} \hlkwb{<-} \hlkwd{MCMCc4photo}\hlstd{(curve1[,}\hlnum{3}\hlopt{:}\hlnum{6}\hlstd{],} \hlkwc{scale}\hlstd{=}\hlnum{1.5}\hlstd{)}
\hlstd{op.mc2} \hlkwb{<-} \hlkwd{MCMCc4photo}\hlstd{(curve1[,}\hlnum{3}\hlopt{:}\hlnum{6}\hlstd{],} \hlkwc{scale}\hlstd{=}\hlnum{1.5}\hlstd{)}
\hlstd{op.mc1}
\hlkwd{plot}\hlstd{(op.mc1, op.mc2,} \hlkwc{plot.kind}\hlstd{=}\hlstr{"density"}\hlstd{,} \hlkwc{burnin}\hlstd{=}\hlnum{1e4}\hlstd{)}
\hlkwd{plot}\hlstd{(op.mc1,} \hlkwc{plot.kind}\hlstd{=}\hlstr{"density"}\hlstd{,} \hlkwc{prior}\hlstd{=}\hlnum{TRUE}\hlstd{,} \hlkwc{burnin}\hlstd{=}\hlnum{1e4}\hlstd{)}
\end{alltt}
\end{kframe}
\end{knitrout}

\code{prior=TRUE} plots the prior along side the results from the MCMC run.  In
this case the prior is very diffuse, but tighter priors would affect the
results. If more stringent priors are set the results will tend to be closer to
the prior distributions (the prior distributions are the black lines in the
graph). In many cases a Bayesian approach is more reasonable as fitting
individual curves in isolation can lead to values outside the range of
reasonable results. The influence of the prior will lead the estimation closer
to 'known' values for a species.

\begin{knitrout}
\definecolor{shadecolor}{rgb}{0.969, 0.969, 0.969}\color{fgcolor}\begin{kframe}
\begin{alltt}
\hlstd{op.mc1} \hlkwb{<-} \hlkwd{MCMCc4photo}\hlstd{(curve1[,}\hlnum{3}\hlopt{:}\hlnum{6}\hlstd{],} \hlkwc{scale}\hlstd{=}\hlnum{1.5}\hlstd{,} \hlkwc{prior}\hlstd{=}\hlkwd{c}\hlstd{(}\hlnum{20}\hlstd{,} \hlnum{1}\hlstd{,} \hlnum{0.045}\hlstd{,} \hlnum{0.0025}\hlstd{))}
\hlkwd{plot}\hlstd{(op.mc1,} \hlkwc{plot.kind} \hlstd{=} \hlstr{"density"}\hlstd{,} \hlkwc{prior} \hlstd{=} \hlnum{TRUE}\hlstd{,} \hlkwc{burnin}\hlstd{=}\hlnum{1e3}\hlstd{,} \hlkwc{lwd}\hlstd{=}\hlnum{2}\hlstd{)}
\end{alltt}
\end{kframe}
\end{knitrout}

TODO: I haven't implemented the possibility of fitting Rd using the Bayesian
approach. This is something I'd like to implement soon. Two other additions
would be including the correlation between parameters and the estimation of the
residual variance. These last features are low priority as the function as it is
should work for most purposes.

To be complete there is even the option of using \code{nls}.

\begin{knitrout}
\definecolor{shadecolor}{rgb}{0.969, 0.969, 0.969}\color{fgcolor}\begin{kframe}
\begin{alltt}
\hlstd{c4photo2} \hlkwb{<-} \hlkwa{function}\hlstd{(}\hlkwc{A}\hlstd{,}\hlkwc{T}\hlstd{,}\hlkwc{RH}\hlstd{,} \hlkwc{vmax}\hlstd{=}\hlnum{39}\hlstd{,} \hlkwc{alpha}\hlstd{=}\hlnum{0.04}\hlstd{)\{}
    \hlstd{res} \hlkwb{<-} \hlkwd{c4photo}\hlstd{(A,T,RH,} \hlkwc{vmax}\hlstd{=vmax,} \hlkwc{alpha}\hlstd{=alpha)}\hlopt{$}\hlstd{assimilation_rate}
    \hlstd{res}
\hlstd{\}}
\hlstd{fit} \hlkwb{<-} \hlkwd{nls}\hlstd{(A} \hlopt{~} \hlkwd{c4photo2}\hlstd{(PARi, Tleaf, RH_S, vmax, alpha),}
           \hlkwc{start}\hlstd{=}\hlkwd{list}\hlstd{(}\hlkwc{vmax}\hlstd{=}\hlnum{39}\hlstd{,} \hlkwc{alpha}\hlstd{=}\hlnum{0.04}\hlstd{),}
           \hlkwc{data} \hlstd{= curve1)}
\end{alltt}
\end{kframe}
\end{knitrout}

TODO: run an example using \code{nlme}. Until I get this to work, the best
option is to do a linear mixed model analysis using \code{lme}.

This concludes the section about estimating photosynthesis parameters in the
context of simulating biomass crops. Although this is the first step, and it is
important, there are many other aspects that influence the simulation of biomass,
transpiration, etc. 

The previous functions are relevant for leaf-level
photosynthesis. Scaling up to the canopy level is not trivial since it
requires developing a light macro environment which simulates the
partitioning between direct and diffuse radiation (see function
\code{lightME}). The function \code{sunML} is used to predict the
proportion of light for each layer of a multiple layered canopy. 

TODO
\begin{itemize}
  \item include an example using \code{c3photo}
  \item Discuss meaning and relationship among parameters
\end{itemize}

\section{Carbon and Water: Canopy Photosynthesis and Transpiration}

The function \code{CanA} integrates the previous functions
to simulate canopy CO$_2$ assimilation for a complete canopy. This
function also simulates transpiration based on Penman-Monteith, Penman
and Priestly. 

The \code{CanA} function is designed to run at an hourly timestep. The
inputs should all be of length 1. As with other canopy models the
canopy is discretized in layers and each layer has unique conditions
in terms of light levels, leaf temperature, wind and relative
humidity. See \code{?CanA} for details.

\begin{knitrout}
\definecolor{shadecolor}{rgb}{0.969, 0.969, 0.969}\color{fgcolor}\begin{kframe}
\begin{alltt}
\hlstd{nlay} \hlkwb{<-} \hlnum{8}
\hlstd{res} \hlkwb{<-} \hlkwd{CanA}\hlstd{(}\hlkwc{lai}\hlstd{=}\hlnum{3}\hlstd{,} \hlkwc{doy}\hlstd{=}\hlnum{200}\hlstd{,} \hlkwc{hr}\hlstd{=}\hlnum{12}\hlstd{,} \hlkwc{solar}\hlstd{=}\hlnum{1500}\hlstd{,} \hlkwc{temp}\hlstd{=}\hlnum{25}\hlstd{,} \hlkwc{rh}\hlstd{=}\hlnum{0.7}\hlstd{,}
            \hlkwc{windspeed}\hlstd{=}\hlnum{2}\hlstd{,} \hlkwc{nlayers}\hlstd{=nlay)}
\end{alltt}
\end{kframe}
\end{knitrout}

The distribution of leaves in the sun and in the shade is an important
characteristic of a canopy and the architecture can be modified mainly
by changing the \code{chi.l} parameter which represents the ratio of
horizontal leaf projections to vertical leaf projections. 

\begin{knitrout}
\definecolor{shadecolor}{rgb}{0.969, 0.969, 0.969}\color{fgcolor}\begin{kframe}
\begin{alltt}
\hlkwd{apply}\hlstd{(res}\hlopt{$}\hlstd{LayMat[,}\hlnum{3}\hlopt{:}\hlnum{4}\hlstd{],} \hlnum{2}\hlstd{, sum)}
\end{alltt}
\end{kframe}
\end{knitrout}

In this example, 1.35 \si{m^2} of leaf are in the sun and 1.65 of
\si{m^2} are in the shade for a total of 3 LAI.

Next we can look in detail about the properties of the canopy by layer



Some important assumption of the multi-layer canopy model are

\begin{itemize}
\item it distributes the LAI equally among layers, this is not
  necessarily a realistic assumption
\item relative humidity increases with canopy depth which causes
  stomatal conductance to increase as well.
\item by default photosynthetic parameters are constant in the profile
  but it is possible to make them depend on a profile of N
  concentration (see argument \code{lnControl})
\end{itemize}

\subsection{Parameters to adjust}

Which parameters are relevant at the canopy level? Of course
photosythetic parameters are important but these were discussed before
so they are assume to be reasonable here. LAI is a very important
input to this function so it is not really an adjustable parameter.

\begin{itemize}
\item \code{nlayers} The number of layers has an effect on many of the
  results. This can be modified by the user if there is a good
  rationale for doing it. It is possible that taller canopies benefit
  more from having multiple layers and shorter canopies benefit less.
\item \code{kd} extinction coefficient for diffuse light. Although this can be
  calculated it is not at this point.
\item \code{chi.l} is the ratio of horizontal to vertical projection
  of leaf area. Lower than 1 values for more erect canopies and less
  than 1 for canopies with higher proportion of flat leaves.
\item \code{leafwidth} average leaf width. 
\item \code{heightFactor} factor relating LAI to height. Adjust it to
  match reasonable height for a crop.
\end{itemize}

\subsection{Water: Calculation of Canopy Transpiration}

\code{CanA} simulates transpiration using Priestly (driven by solar
radiation and temperature), Penman-Potential (adjusted for the aerodynamic
component) and Penman-Monteith (adjusted for the aerodynamic plus
stomatal component).



At the moment Penman over estimates transpiration, Priestly and
Penman-Monteith seem to be giving reasonable answers.

\subsection{Water: Effect of Ball-Berry slope parameter}

The slope of the Ball-Berry model can have a significant effect on the
results, but only for the Penman-Monteith method. The parameters for
Ball-Berry however should be set from previous literature data or from
analysis of gas exchange measurements. The purpose of this is to show
that it has a large effect. It is supplied by the \code{photoParms}
function. This increases transpiration in the Penman-Monteith model
but only up to a point.  Priestly and Penman are almost always higher
than Penman-Monteith.





\subsection{Water: Effect of stress on diurnal transpiration}

Another significant component that will affect transpiration during a
day is the level of water stress the plant is experiencing.





TODO
\begin{itemize}
\item Include an example in which I see the effect of canopy height on
  diurnal transpiration.
\item Include an example in which I see the effect of changing chi.l
\end{itemize}

\section{Carbon and Water: Biomass Crop Simulation}

When the interest is to perform a simulation for a whole growing
season, the function \code{BioGro} can be used. This function has as
minimum input weather data for the whole year (365 days) at hourly
time steps. The data can be generated using the \code{weach} function
from daily data. One of the built-in datasets is \code{cmi04} which is
weather data for Champaign, IL for 2004.

\begin{knitrout}
\definecolor{shadecolor}{rgb}{0.969, 0.969, 0.969}\color{fgcolor}\begin{kframe}
\begin{alltt}
\hlkwd{data}\hlstd{(cmi04)}
\hlkwd{summary}\hlstd{(cmi04)}
\hlstd{soilP} \hlkwb{<-} \hlkwd{soilParms}\hlstd{(}\hlkwc{wsFun}\hlstd{=}\hlstr{'linear'}\hlstd{)}
\hlstd{res} \hlkwb{<-} \hlkwd{BioGro}\hlstd{(cmi04,} \hlkwc{soilControl}\hlstd{=soilP)}
\hlkwd{plot}\hlstd{(res)}
\hlkwd{plot}\hlstd{(res,} \hlkwc{plot.kind}\hlstd{=}\hlstr{"SW"}\hlstd{)}
\hlkwd{plot}\hlstd{(res,} \hlkwc{plot.kind}\hlstd{=}\hlstr{"ET"}\hlstd{)}
\hlkwd{plot}\hlstd{(res,} \hlkwc{plot.kind}\hlstd{=}\hlstr{"cumET"}\hlstd{)}
\hlkwd{plot}\hlstd{(res,} \hlkwc{plot.kind}\hlstd{=}\hlstr{"stress"}\hlstd{)}
\hlkwd{names}\hlstd{(res)}
\end{alltt}
\end{kframe}
\end{knitrout}

The last command \code{names(res)} shows the list of objects available
for further manipulation. 

\subsection{Water: Calculation of EvapoTranspiration}

As in the \code{CanA} function transpiration can be calculated using
Priestly, Penman-Potential or Penman-Monteith.



For Priestly and Penman-Potential the calculation of stomatal conductance does
not affect the simulation, but for Penman-Monteith it does.



Penman-Moneith does not quite reach Priestly in this case because of
stress. It is possible to perform simulations assuming that there is
no stress but this will lead to results which approach
Penman-Potential. When the no-stress option is selected the crop
transpires freely unconstrained by soil water availability. This is
not realistic but it is useful for testing and understanding
transpiration processes.



\subsection{Water: Balance for a growing season}

The example below shows a water balance which consists of taking into
account precipitation, evapotranspiration, drainage, runoff and change
in water storage.

\[ P - (ET + RO + DR + \Delta \Theta) = 0 \]

where 

$P$ is precipitation (mm)
$ET$ is evapotranspiration (mm)
$RO$ is runoff (mm)
$DR$ is drainage (mm)
$\Delta \Theta$ is change in soil water storage.

\begin{knitrout}
\definecolor{shadecolor}{rgb}{0.969, 0.969, 0.969}\color{fgcolor}\begin{kframe}
\begin{alltt}
\hlcom{## Simple water budget}
\hlcom{## P - ET + RO + DR + DeltaTheta = 0}
\hlkwd{data}\hlstd{(cmi04)}
\hlstd{day1} \hlkwb{<-} \hlnum{100}
\hlstd{dayn} \hlkwb{<-} \hlnum{270}
\hlstd{cmi04.s} \hlkwb{<-} \hlkwd{subset}\hlstd{(cmi04, doy} \hlopt{>} \hlnum{99} \hlopt{&} \hlstd{doy} \hlopt{<} \hlnum{271}\hlstd{)}
\hlstd{P} \hlkwb{<-} \hlkwd{sum}\hlstd{(cmi04.s}\hlopt{$}\hlstd{precip)} \hlcom{## in mm}
\hlstd{iwc} \hlkwb{<-} \hlnum{0.29}
\hlstd{soildepth} \hlkwb{<-} \hlnum{2}
\hlstd{soilP} \hlkwb{<-} \hlkwd{soilParms}\hlstd{(}\hlkwc{iWatCont}\hlstd{=iwc,} \hlkwc{soilDepth}\hlstd{=soildepth,} \hlkwc{soilLayers}\hlstd{=}\hlnum{1}\hlstd{)}
\hlstd{res} \hlkwb{<-} \hlkwd{BioGro}\hlstd{(cmi04,} \hlkwc{day1}\hlstd{=}\hlnum{100}\hlstd{,} \hlkwc{dayn}\hlstd{=}\hlnum{270}\hlstd{,} \hlkwc{soilControl} \hlstd{= soilP)}
\hlstd{et} \hlkwb{<-} \hlstd{res}\hlopt{$}\hlstd{CanopyTrans} \hlopt{+} \hlstd{res}\hlopt{$}\hlstd{SoilEvaporation}
\hlstd{ET} \hlkwb{<-} \hlkwd{sum}\hlstd{(et)} \hlopt{*} \hlstd{(}\hlnum{1}\hlopt{/}\hlnum{0.9982}\hlstd{)} \hlopt{*}\hlnum{0.1}
\hlcom{## in mm, 0.9982 accounts for density of water}
\hlstd{RO} \hlkwb{<-} \hlkwd{sum}\hlstd{(res}\hlopt{$}\hlstd{Runoff)} \hlcom{## in mm}
\hlstd{DR} \hlkwb{<-} \hlkwd{sum}\hlstd{(res}\hlopt{$}\hlstd{Drainage)} \hlcom{## in mm }
\hlstd{iTheta} \hlkwb{<-} \hlstd{iwc} \hlopt{*} \hlstd{soildepth}
\hlstd{fswc} \hlkwb{<-} \hlstd{res}\hlopt{$}\hlstd{SoilWatCont[}\hlkwd{length}\hlstd{(res}\hlopt{$}\hlstd{SoilWatCont)]}
\hlstd{fswc}
\hlstd{fTheta} \hlkwb{<-} \hlstd{fswc} \hlopt{*} \hlstd{soildepth}
\hlstd{DeltaTheta} \hlkwb{<-} \hlstd{(fTheta} \hlopt{-} \hlstd{iTheta)} \hlopt{*} \hlnum{1e3} \hlcom{## from m to mm}
\hlkwd{cbind}\hlstd{(P, ET, DeltaTheta, RO, DR)}
\hlstd{P} \hlopt{-} \hlstd{(ET} \hlopt{+} \hlstd{DeltaTheta} \hlopt{+} \hlstd{RO} \hlopt{+} \hlstd{DR)}
\end{alltt}
\end{kframe}
\end{knitrout}

\end{document}

